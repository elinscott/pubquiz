\documentclass[xcolor=table,aspectratio=169]{beamer}
\usetheme{Madrid}
\usepackage{adjustbox}
%\usetheme{metropolis}
\usepackage[T1]{fontenc}
\usepackage{graphicx}

\definecolor{twitter_blue}{HTML}{1da1f2}
% Definitions of colours used in seaborn for use in latex
\definecolor{seaborn_bg_grey}{HTML}{eaeaf2}
\definecolor{seaborn_bg_grey_dark}{HTML}{d2d2d9}
\definecolor{seaborn_bg_grey_darker}{HTML}{a3a3a9}
\definecolor{seaborn_bg_grey_half}{HTML}{f4f4f8}

\definecolor{seaborn_blue}{HTML}{4c72b0}
\definecolor{seaborn_green}{HTML}{55a868}
\definecolor{seaborn_red}{HTML}{c44e52}
\definecolor{seaborn_magenta}{HTML}{8172b2}
\definecolor{seaborn_yellow}{HTML}{ccb974}
\definecolor{seaborn_cyan}{HTML}{64b5cd}

\definecolor{seaborn_muted_blue}{HTML}{4878cf}
\definecolor{seaborn_muted_green}{HTML}{6acc65}
\definecolor{seaborn_muted_red}{HTML}{d65f5f}
\definecolor{seaborn_muted_magenta}{HTML}{b47cc7}
\definecolor{seaborn_muted_yellow}{HTML}{c4ad66}
\definecolor{seaborn_muted_cyan}{HTML}{77bedb}

\definecolor{seaborn_pastel_blue}{HTML}{92c6ff}
\definecolor{seaborn_pastel_green}{HTML}{97f0aa}
\definecolor{seaborn_pastel_red}{HTML}{ff9f9a}
\definecolor{seaborn_pastel_magenta}{HTML}{d0bbff}
\definecolor{seaborn_pastel_yellow}{HTML}{fffea3}
\definecolor{seaborn_pastel_cyan}{HTML}{b0e0e6}

\definecolor{seaborn_bright_blue}{HTML}{003fff}
\definecolor{seaborn_bright_green}{HTML}{03ed3a}
\definecolor{seaborn_bright_red}{HTML}{e8000b}
\definecolor{seaborn_bright_magenta}{HTML}{8a2be2}
\definecolor{seaborn_bright_yellow}{HTML}{ffc400}
\definecolor{seaborn_bright_cyan}{HTML}{00d7ff}

\definecolor{seaborn_dark_blue}{HTML}{001c7f}
\definecolor{seaborn_dark_green}{HTML}{017517}
\definecolor{seaborn_dark_red}{HTML}{8c0900}
\definecolor{seaborn_dark_magenta}{HTML}{7600a1}
\definecolor{seaborn_dark_yellow}{HTML}{b8860b}
\definecolor{seaborn_dark_cyan}{HTML}{006374}

\definecolor{seaborn_colorblind_blue}{HTML}{0072b2}
\definecolor{seaborn_colorblind_green}{HTML}{009e73}
\definecolor{seaborn_colorblind_red}{HTML}{d55e00}
\definecolor{seaborn_colorblind_magenta}{HTML}{cc79a7}
\definecolor{seaborn_colorblind_yellow}{HTML}{f0e442}
\definecolor{seaborn_colorblind_cyan}{HTML}{56b4e9}



% "blindfootcite" is the equivalent of "footcite" except the number marker does not appear
\newcommand\blfootcite[1]{%
  \begingroup
  \renewcommand\thefootnote{}\footnote{\hspace{-4ex}\cite{#1}}%
  \addtocounter{footnote}{-1}%
  \endgroup
}
\setbeamerfont{footnote}{size=\scriptsize}
\renewcommand\footnoterule{\kern-3pt \color{seaborn_bg_grey_darker}\hrule width \textwidth height 0.4pt \color{black} \kern 2.6pt}

\beamertemplatenavigationsymbolsempty

% \renewcommand*{\bibfont}{\tiny}
\usepackage{amssymb}
\usepackage{epsfig}
\usepackage{psfrag}
\usepackage{wrapfig}
\usepackage{graphicx}
\usepackage{color}
\usepackage[table]{xcolor}
\usepackage{amsmath}
\usepackage{multimedia}
\usepackage{subcaption}
%\usepackage{style}
\usepackage{verbatim}
\usepackage{multicol}
\usepackage[table]{xcolor}
\usepackage{tabularx}
% Tikz
\usepackage{tikz}
\usetikzlibrary{positioning,shapes,arrows,backgrounds,fit,calc,external,trees,tikzmark}
% \tikzexternalize[prefix=tikzfigures/]
\tikzstyle{dummy} = []
\tikzstyle{line} = [draw, thick, -latex']
\tikzstyle{headless_line} = [draw, thick, -]
\tikzstyle{default}    = [rectangle, text centered, rounded corners, text=black, font=\sffamily\footnotesize, align=center]
\tikzstyle{default_text}    = [rectangle, text width=10cm, text=black,anchor=north west, font=\sffamily]
\tikzstyle{boxwhite} = [default, fill=white, rounded corners=0.1cm]
\tikzstyle{cp}    = [default, fill=seaborn_blue, text=white, text width=2.8cm, minimum height=0.5cm]
\tikzstyle{pw}    = [cp, fill=seaborn_green]
\tikzstyle{wannier90}    = [cp, fill=seaborn_cyan]
\tikzstyle{bespoke}    = [cp, fill=seaborn_magenta]
\tikzstyle{observable}    = [cp, fill=seaborn_red]
\tikzset{
  -|-/.style={
    to path={
      (\tikztostart) -| ($(\tikztostart)!#1!(\tikztotarget)$) |- (\tikztotarget)
      \tikztonodes
    }
  },
  -|-/.default=0.5,
  |-|/.style={
    to path={
      (\tikztostart) |- ($(\tikztostart)!#1!(\tikztotarget)$) -| (\tikztotarget)
      \tikztonodes
    }
  },
  |-|/.default=0.5,
}

\newlength{\myyshift}
\setlength{\myyshift}{0.05cm}

\usepackage{lipsum}
\usetikzlibrary{calc}
\newlength{\myfigscale}
\setlength{\myfigscale}{0.3cm}
\usepackage{smartdiagram}
\usesmartdiagramlibrary{additions}
\usepackage{multicol}
\usepackage{helvet}
% \usepackage{sansmath}
% \sansmath
\usepackage{cancel} % for \cancel
\usepackage[normalem]{ulem} % for sout (strike out)
\usepackage{tcolorbox}
\tcbuselibrary{skins,hooks}
\tcbset{colframe=structure,fonttitle=\bfseries,beamer, clip upper, boxsep=0pt, sharp corners=all, no shadow, left skip=0pt, right skip=0pt, coltext=white}

% For electron orbital diagrams
\usepackage{tikzorbital}
% Changing defaults
\pgfkeys{tikzorbital/drawLevel/width = 0.666666}
\pgfkeys{tikzorbital/drawLevel/style = {line width = 1pt, color = black!80, line cap = round}}
\pgfkeys{tikzorbital/drawLevel/spinlength = 0.666666}
\pgfkeys{tikzorbital/drawLevel/spinstyle = {very thick, color = black!80, -stealth}}

% Definitions of colours used in seaborn for use in latex
\definecolor{seaborn_bg_grey}{HTML}{eaeaf2}
\definecolor{seaborn_bg_grey_dark}{HTML}{d2d2d9}
\definecolor{seaborn_bg_grey_darker}{HTML}{a3a3a9}
\definecolor{seaborn_bg_grey_half}{HTML}{f4f4f8}

\definecolor{seaborn_blue}{HTML}{4c72b0}
\definecolor{seaborn_green}{HTML}{55a868}
\definecolor{seaborn_red}{HTML}{c44e52}
\definecolor{seaborn_magenta}{HTML}{8172b2}
\definecolor{seaborn_yellow}{HTML}{ccb974}
\definecolor{seaborn_cyan}{HTML}{64b5cd}

\definecolor{seaborn_muted_blue}{HTML}{4878cf}
\definecolor{seaborn_muted_green}{HTML}{6acc65}
\definecolor{seaborn_muted_red}{HTML}{d65f5f}
\definecolor{seaborn_muted_magenta}{HTML}{b47cc7}
\definecolor{seaborn_muted_yellow}{HTML}{c4ad66}
\definecolor{seaborn_muted_cyan}{HTML}{77bedb}

\definecolor{seaborn_pastel_blue}{HTML}{92c6ff}
\definecolor{seaborn_pastel_green}{HTML}{97f0aa}
\definecolor{seaborn_pastel_red}{HTML}{ff9f9a}
\definecolor{seaborn_pastel_magenta}{HTML}{d0bbff}
\definecolor{seaborn_pastel_yellow}{HTML}{fffea3}
\definecolor{seaborn_pastel_cyan}{HTML}{b0e0e6}

\definecolor{seaborn_bright_blue}{HTML}{003fff}
\definecolor{seaborn_bright_green}{HTML}{03ed3a}
\definecolor{seaborn_bright_red}{HTML}{e8000b}
\definecolor{seaborn_bright_magenta}{HTML}{8a2be2}
\definecolor{seaborn_bright_yellow}{HTML}{ffc400}
\definecolor{seaborn_bright_cyan}{HTML}{00d7ff}

\definecolor{seaborn_dark_blue}{HTML}{001c7f}
\definecolor{seaborn_dark_green}{HTML}{017517}
\definecolor{seaborn_dark_red}{HTML}{8c0900}
\definecolor{seaborn_dark_magenta}{HTML}{7600a1}
\definecolor{seaborn_dark_yellow}{HTML}{b8860b}
\definecolor{seaborn_dark_cyan}{HTML}{006374}

\definecolor{seaborn_colorblind_blue}{HTML}{0072b2}
\definecolor{seaborn_colorblind_green}{HTML}{009e73}
\definecolor{seaborn_colorblind_red}{HTML}{d55e00}
\definecolor{seaborn_colorblind_magenta}{HTML}{cc79a7}
\definecolor{seaborn_colorblind_yellow}{HTML}{f0e442}
\definecolor{seaborn_colorblind_cyan}{HTML}{56b4e9}



% For tikz diagrams with nodes appearing on each slide
\tikzset{
  invisible/.style={opacity=0},
  visible on/.style={alt={#1{}{invisible}}},
  alt/.code args={<#1>#2#3}{%
    \alt<#1>{\pgfkeysalso{#2}}{\pgfkeysalso{#3}} % \pgfkeysalso doesn't change the path
  },
}

\usepackage{array}
\usepackage{multirow}
% \newcolumntype{L}[1]{>{\raggedright\let\newline\\\arraybackslash\hspace{0pt}}m{#1}}
% \newcolumntype{C}[1]{>{\centering\let\newline\\\arraybackslash\hspace{0pt}}m{#1}}
% \newcolumntype{R}[1]{>{\raggedleft\let\newline\\\arraybackslash\hspace{0pt}}m{#1}}
\newcolumntype{L}{>{\raggedright\arraybackslash}X}
\newcolumntype{C}{>{\centering\arraybackslash}X}
\newcolumntype{R}{>{\raggedleft\arraybackslash}X}

% For checklist
%\usepackage{enumitem}
%\newlist{todolist}{itemize}{2}
%\setlist[todolist]{label=$\square$}
\usepackage{pifont}
\newcommand{\cmark}{\ding{51}}%
\newcommand{\xmark}{\ding{55}}%
\newcommand{\done}{\rlap{$\square$}{\raisebox{2pt}{\large\hspace{1pt}\cmark}}%
\hspace{-2.5pt}}
\newcommand{\wontfix}{\rlap{$\square$}{\large\hspace{1pt}\xmark}}

\newcommand{\bra}[1]{\langle #1|}
\newcommand{\braket}[2]{\langle #1|#2\rangle}
\newcommand{\braopket}[3]{\langle #1|#2|#3\rangle}
\newcommand{\ket}[1]{|#1\rangle}
\newcommand{\nline}{\nonumber \\}
\newcommand{\Trace}{\mathsf{Tr}}

\renewcommand{\ttdefault}{pcr} % enables bold fixed width font
\numberwithin{equation}{section}
% \usefonttheme{professionalfonts}
%\usefonttheme[stillsansseriflarge,stillsansserifsmall]{serif}
\usepackage{siunitx,booktabs}
% \AtBeginDocument{\sisetup{math-rm=\mathsf, text-rm=\sffamily}}
\AtBeginEnvironment{frame}{\setcounter{footnote}{0}}

\newlength{\myimscale}


% For code blocks in latex
% Taken from https://github.com/daveyarwood/gruvbox-pygments
% N.B.
%  - frame must have [fragile]
%  - use \begin{onlyenv} not \only
%  - after a lot of mucking around, I created gruvbox_plain as another style
%    that exclusively uses gruvbox's bg and fg with no syntax highlighting
%  - use [autogobble] to remove leading indentations

\usepackage{minted}
\usemintedstyle{gruvbox-dark}
\definecolor{gruvbox_dark_bg}{HTML}{282828}
\definecolor{gruvbox_fg}{HTML}{ebdbb2}
\definecolor{kgrey}{HTML}{2b2828}
\setminted[python]{bgcolor=gruvbox_dark_bg}
\setminted[json]{bgcolor=gruvbox_dark_bg}
\setminted[shell-session]{style=gruvbox_plain, bgcolor=gruvbox_dark_bg}

% \lstset{breaklines,breakatwhitespace,breakautoindent=false,showstringspaces=false}
% \lstset{keywordstyle=\color{purple}}
% \lstset{identifierstyle=\color{blue}}
% \lstset{basicstyle=\fontfamily{pcr}\fontsize{9pt}{9pt}\selectfont}
% %\lstset{numbers=left, numberstyle=\tiny, stepnumber=1, numbersep=5pt}
% \lstset{linewidth=4.9in,xleftmargin=10pt}

\setbeamercolor{frametitle}{bg=kgrey,fg=white}
\setbeamerfont{normal text}{family=helvet}
\setbeamerfont{local structure}{family=helvet}

\setbeamercolor*{author in head/foot}{bg=seaborn_blue}
\setbeamercolor*{logo in head/foot}{bg=seaborn_blue,fg=white}
\setbeamercolor*{title in head/foot}{bg=seaborn_blue,fg=white}
\setbeamercolor*{date in head/foot}{bg=seaborn_blue,fg=white}
\setbeamercolor{title}{bg=seaborn_blue}
\setbeamercolor{under headline}{bg=seaborn_red}
\setbeamercolor{footline}{bg=seaborn_blue}
\setbeamercolor{caption name}{fg=seaborn_blue}
\setbeamercolor{block title}{bg=kgrey,fg=white}
\setbeamercolor{block body}{bg=seaborn_bg_grey,fg=black}

% Footnote style and colour
% No line over footnote
\setbeamercolor{footnote}{fg=seaborn_bg_grey_darker}

\setbeamertemplate{enumerate items}[default]
\setbeamertemplate{blocks}[default]
\setbeamertemplate{itemize items}{\normalsize $\bullet$}
\setbeamercolor{description item}{fg=seaborn_blue}
\setbeamercolor{enumerate item}{fg=seaborn_blue}
\setbeamercolor{itemize item}{fg=seaborn_blue}
\setbeamercolor{itemize subitem}{fg=seaborn_blue}
\setbeamercolor{itemize subsubitem}{fg=seaborn_blue}
\setbeamercolor*{bibliography entry title}{fg=seaborn_bg_grey_darker}
\setbeamercolor*{bibliography entry author}{fg=seaborn_bg_grey_darker}
\setbeamercolor*{bibliography entry location}{fg=seaborn_bg_grey_darker}
\setbeamercolor*{bibliography entry note}{fg=seaborn_bg_grey_darker}
% and kill the abominable icon
\setbeamertemplate{bibliography item}[text]

\setbeamerfont*{title in head/foot}{size=\small}
\setbeamerfont*{date in head/foot}{size=\small}
\setbeamerfont*{institute}{size=\Large}

\setbeamertemplate{frametitle}
{
  \leavevmode%
  \vspace{-20pt}
  \begin{beamercolorbox}[wd=\paperwidth,ht=1cm]{frametitle}
   \hspace{0.115em}
   \vphantom{P/p} \bf \insertframetitle \vspace{0.2cm}
   \end{beamercolorbox}%
  %  \vskip-0.6cm%
  % \begin{beamercolorbox}[wd=\paperwidth,ht=0.5ex]{under headline}%
  %   \end{beamercolorbox}%
	
}

\newcommand{\insertframeinfo}{| \insertframenumber/\inserttotalframenumber}
\newcommand{\backupbegin}{
   \newcounter{finalframe}
   \setcounter{finalframe}{\value{framenumber}}
   \renewcommand{\insertframeinfo}{}
}
\newcommand{\backupend}{
   \setcounter{framenumber}{\value{finalframe}}
}


\setbeamertemplate{frametitle}
{
%   \vspace{-1pt}
%   \begin{beamercolorbox}[wd=\paperwidth,ht=0.8cm]{frametitle}
%    \hspace{0.05em}
%    \begin{minipage}{0.8\textwidth}
%      \bf \insertframetitle\vphantom{Pp}
%    \end{minipage}
%    \hfill
%    \vspace{0.125cm}
%   \end{beamercolorbox}%
}

\setbeamertemplate{title page}
{
  \leavevmode%
  \vbox{%
  \vspace{-1.6ex}%
  \noindent\begin{tcolorbox}[enhanced,watermark graphics=photos/EPFL-Leman-vue-aerienne-1536x864.jpg, width=\paperwidth, height=0.575\paperwidth, watermark zoom=1.25, grow to left by=0.035\paperwidth, frame hidden]

  \vspace{7.5em}
  \begin{center} 
  \huge
  \textbf{\inserttitle}

  \end{center}
  \end{tcolorbox}

  \vspace{-2em}
  \begin{tcolorbox}[width=\paperwidth, enhanced, colback=kgrey, grow to left by=0.035\paperwidth,]
  \begin{center}
  \footnotesize \bf \insertauthor\quad | \quad \insertdate    
  \end{center}
  %  \end{flushright}
  \end{tcolorbox}
  }


	
}
%\setbeamerfont{frametitle}{series=\bfseries}
\setbeamertemplate{footline}
{
}

% Title slide %%%%%%%%%%%%%%%%%%%%%%%%%%%%%%%%%%%%%%%%%%%%%%%%%%%%%%%%%%%%%%%%%%%%%%%%%%%%%%%%%%%
% \title[Beamer Intro]{\noindent Strongly correlated biological systems\\\Large DFT and beyond}
\title{THEOS pub quiz}
\author{Edward Linscott}
\institute{EPFL}
\date{27 Oct 2023}
